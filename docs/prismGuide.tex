% Options for packages loaded elsewhere
\PassOptionsToPackage{unicode}{hyperref}
\PassOptionsToPackage{hyphens}{url}
%
\documentclass[
]{book}
\usepackage{lmodern}
\usepackage{amssymb,amsmath}
\usepackage{ifxetex,ifluatex}
\ifnum 0\ifxetex 1\fi\ifluatex 1\fi=0 % if pdftex
  \usepackage[T1]{fontenc}
  \usepackage[utf8]{inputenc}
  \usepackage{textcomp} % provide euro and other symbols
\else % if luatex or xetex
  \usepackage{unicode-math}
  \defaultfontfeatures{Scale=MatchLowercase}
  \defaultfontfeatures[\rmfamily]{Ligatures=TeX,Scale=1}
\fi
% Use upquote if available, for straight quotes in verbatim environments
\IfFileExists{upquote.sty}{\usepackage{upquote}}{}
\IfFileExists{microtype.sty}{% use microtype if available
  \usepackage[]{microtype}
  \UseMicrotypeSet[protrusion]{basicmath} % disable protrusion for tt fonts
}{}
\makeatletter
\@ifundefined{KOMAClassName}{% if non-KOMA class
  \IfFileExists{parskip.sty}{%
    \usepackage{parskip}
  }{% else
    \setlength{\parindent}{0pt}
    \setlength{\parskip}{6pt plus 2pt minus 1pt}}
}{% if KOMA class
  \KOMAoptions{parskip=half}}
\makeatother
\usepackage{xcolor}
\IfFileExists{xurl.sty}{\usepackage{xurl}}{} % add URL line breaks if available
\IfFileExists{bookmark.sty}{\usepackage{bookmark}}{\usepackage{hyperref}}
\hypersetup{
  pdftitle={Models on the Peer Models Network},
  pdfauthor={Amin Adibi, Stephanie Harvard, Mohsen Sadatsafavi},
  hidelinks,
  pdfcreator={LaTeX via pandoc}}
\urlstyle{same} % disable monospaced font for URLs
\usepackage{longtable,booktabs}
% Correct order of tables after \paragraph or \subparagraph
\usepackage{etoolbox}
\makeatletter
\patchcmd\longtable{\par}{\if@noskipsec\mbox{}\fi\par}{}{}
\makeatother
% Allow footnotes in longtable head/foot
\IfFileExists{footnotehyper.sty}{\usepackage{footnotehyper}}{\usepackage{footnote}}
\makesavenoteenv{longtable}
\usepackage{graphicx,grffile}
\makeatletter
\def\maxwidth{\ifdim\Gin@nat@width>\linewidth\linewidth\else\Gin@nat@width\fi}
\def\maxheight{\ifdim\Gin@nat@height>\textheight\textheight\else\Gin@nat@height\fi}
\makeatother
% Scale images if necessary, so that they will not overflow the page
% margins by default, and it is still possible to overwrite the defaults
% using explicit options in \includegraphics[width, height, ...]{}
\setkeys{Gin}{width=\maxwidth,height=\maxheight,keepaspectratio}
% Set default figure placement to htbp
\makeatletter
\def\fps@figure{htbp}
\makeatother
\setlength{\emergencystretch}{3em} % prevent overfull lines
\providecommand{\tightlist}{%
  \setlength{\itemsep}{0pt}\setlength{\parskip}{0pt}}
\setcounter{secnumdepth}{5}
\usepackage{booktabs}
\usepackage[]{natbib}
\bibliographystyle{apalike}

\title{Models on the Peer Models Network}
\author{Amin Adibi, Stephanie Harvard, Mohsen Sadatsafavi}
\date{2020-06-19}

\begin{document}
\maketitle

{
\setcounter{tocdepth}{1}
\tableofcontents
}
\hypertarget{introduction}{%
\chapter{Introduction}\label{introduction}}

This user guide includes information about models hosted on the Peer Models Network.

\hypertarget{accept}{%
\chapter{ACCEPT}\label{accept}}

\textbf{Model Name:} Acute COPD Exacerbation Prediction Tool (ACCEPT)

\textbf{Modelling team:} Respiratory Evaluation Sciences Program (RESP), at the Faculty of Pharmaceutical Sciences at the University of British Columbia \url{http://resp.core.ubc.ca}

\textbf{Link to published manuscript, pre-print, or other report:} Adibi A, Sin DD, Safari A, Jonhson KM, Aaron SD, FitzGerald JM, Sadatsafavi M. The Acute COPD Exacerbation Prediction Tool (ACCEPT): a modelling study. The Lancet Respiratory Medicine. Published Online First 2020 March 13th; \href{https://doi.org/10.1016/S2213-2600(19)30397-2}{doi:10.1016/S2213-2600(19)30397-2}

\textbf{Purpose of the model:} To predict probably, rate, and severity of COPD exacerbations within the next year.

\textbf{Outcome measure:} Probability and rate of all and severe exacerbations within the next year.

\textbf{Model companion video(s): }\href{https://www.peermodelsnetwork.com/educational-videos?wix-vod-video-id=679cbb410686401193779a2931731c56\&wix-vod-comp-id=comp-k8q9lys1}{The ACCEPT Model in 90 Seconds}

\textbf{Interview with modeller:} Read the interview with Amin Adibi on Peer Models Network \href{peermodelsnetwork.com/blog}{blog}.

\textbf{Interview with stakeholder(s) or other media coverage:}

Lancet Respiratory Medicine: \href{https://doi.org/10.1016/S2213-2600(20)30049-7}{COPD exacerbations: finally, a more than ACCEPTable risk score}

\hypertarget{how-to-access-the-model}{%
\section{How to Access the Model}\label{how-to-access-the-model}}

ACCEPT is available as a \href{http://resp.core.ubc.ca/ipress/accept}{web app} and an \href{https://cran.r-project.org/package=accept}{R package}. Additionally, users can access ACCEPT on the Peer Model Network's cloud.

Additionally, the \href{http://prism.resp.core.ubc.ca}{Peer Models Network} allows users to access ACCEPT through the cloud through Microsoft Excel, R, Python, JavaScript, Linux bash and any other platform that supports modern APIs.

\textbf{Microsoft Excel}

A MACRO-enabled Excel-file can be used to interact with the model and see the results. To download the PRISM Excel template file for ACCEPT, please refer to the \href{http://resp.core.ubc.ca/ipress/prism}{PRISM model repository}.

\textbf{Cloud Access through R}

User's can access models on the Peer Models Network using the \texttt{peermodels} R package, available on \href{https://github.com/resplab/peermodels}{GitHub}. The following code snippet illustrates how you can run the model for example patients provided in the \texttt{accept} package:

\begin{verbatim}
remotes::install_github (resplab/peermodels)
library(peermodels)
connect_to_model("accept", api_key = YOUR_API_KEY)
input <- get_default_input()
results <- model_run(input)
\end{verbatim}

\textbf{Cloud Access through Python}

\begin{verbatim}
import json
import requests
url = 'https://prism.peermodelsnetwork.com/route/accept/run'
headers = {'x-prism-auth-user': YOUR_API_KEY}
model_run = requests.post(url, headers=headers,
json = {"func":["prism_model_run"],"model_input":[{"ID": "10001","male": 1,"age": 57,"smoker": 0,"oxygen": 0,"statin": 0,"LAMA": 1,"LABA": 1,"ICS": 1,"FEV1": 51,"BMI": 18,"SGRQ": 63,"LastYrExacCount": 2,"LastYrSevExacCount": 1,"randomized_azithromycin": 0,"randomized_statin": 0,"randomized_LAMA": 0,"randomized_LABA": 0,"randomized_ICS": 0, "random_sampling_N" : 1000, "random_distribution_iteration" : 20000, "calculate_CIs" : "TRUE"}]})
print(model_run)
results = json.loads(model_run.text)
print(results)
\end{verbatim}

\textbf{Cloud Access through Linux Bash}

In Ubuntu, you can call the API with \texttt{curl}:

\begin{verbatim}
curl \
-X POST \
-H "x-prism-auth-user: REPLACE_WITH_API_KEY" \
-H "Content-Type: application/json" \
-d '{"func":["prism_model_run"],"model_input":[{"ID": "10001","male": 1,"age": 57,"smoker": 0,"oxygen": 0,"statin": 0,"LAMA": 1,"LABA": 1,"ICS": 1,"FEV1": 51,"BMI": 18,"SGRQ": 63,"LastYrExacCount": 2,"LastYrSevExacCount": 1,"randomized_azithromycin": 0,"randomized_statin": 0,"randomized_LAMA": 0,"randomized_LABA": 0,"randomized_ICS": 0, "random_sampling_N" : 1000, "random_distribution_iteration" : 20000, "calculate_CIs" : "TRUE"}]}' \
https://prism.peermodelsnetwork.com/route/accept/run
\end{verbatim}

  \bibliography{book.bib,packages.bib}

\end{document}
